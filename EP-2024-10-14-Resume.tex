\documentclass[10pt, letterpaper]{article}

% Packages:
\usepackage[
    ignoreheadfoot, % set margins without considering header and footer
    top=2 cm, % seperation between body and page edge from the top
    bottom=2 cm, % seperation between body and page edge from the bottom
    left=2 cm, % seperation between body and page edge from the left
    right=2 cm, % seperation between body and page edge from the right
    footskip=1.0 cm, % seperation between body and footer
    % showframe % for debugging 
]{geometry} % for adjusting page geometry
\usepackage{titlesec} % for customizing section titles
\usepackage{tabularx} % for making tables with fixed width columns
\usepackage{array} % tabularx requires this
\usepackage[dvipsnames]{xcolor} % for coloring text
\definecolor{primaryColor}{RGB}{0, 79, 144} % define primary color
\usepackage{enumitem} % for customizing lists
\usepackage{fontawesome5} % for using icons
\usepackage{amsmath} % for math
\usepackage[
    pdftitle={Elijah Parker's CV},
    pdfauthor={Elijah Parker},
    pdfcreator={LaTeX with RenderCV},
    colorlinks=true,
    urlcolor=primaryColor
]{hyperref} % for links, metadata and bookmarks
\usepackage[pscoord]{eso-pic} % for floating text on the page
\usepackage{calc} % for calculating lengths
\usepackage{bookmark} % for bookmarks
\usepackage{lastpage} % for getting the total number of pages
\usepackage{changepage} % for one column entries (adjustwidth environment)
\usepackage{paracol} % for two and three column entries
\usepackage{ifthen} % for conditional statements
\usepackage{needspace} % for avoiding page brake right after the section title
\usepackage{iftex} % check if engine is pdflatex, xetex or luatex

% Ensure that generate pdf is machine readable/ATS parsable:
\ifPDFTeX
    \input{glyphtounicode}
    \pdfgentounicode=1
    % \usepackage[T1]{fontenc} % this breaks sb2nov
    \usepackage[utf8]{inputenc}
    \usepackage{lmodern}
\fi



% Some settings:
\AtBeginEnvironment{adjustwidth}{\partopsep0pt} % remove space before adjustwidth environment
\pagestyle{empty} % no header or footer
\setcounter{secnumdepth}{0} % no section numbering
\setlength{\parindent}{0pt} % no indentation
\setlength{\topskip}{0pt} % no top skip
\setlength{\columnsep}{0cm} % set column seperation
\makeatletter
\let\ps@customFooterStyle\ps@plain % Copy the plain style to customFooterStyle
\patchcmd{\ps@customFooterStyle}{\thepage}{
    \color{gray}\textit{\small Elijah Parker - Page \thepage{} of \pageref*{LastPage}}
}{}{} % replace number by desired string
\makeatother
\pagestyle{customFooterStyle}

\titleformat{\section}{\needspace{4\baselineskip}\bfseries\large}{}{0pt}{}[\vspace{1pt}\titlerule]

\titlespacing{\section}{
    % left space:
    -1pt
}{
    % top space:
    0.3 cm
}{
    % bottom space:
    0.2 cm
} % section title spacing

\renewcommand\labelitemi{$\circ$} % custom bullet points
\newenvironment{highlights}{
    \begin{itemize}[
        topsep=0.10 cm,
        parsep=0.10 cm,
        partopsep=0pt,
        itemsep=0pt,
        leftmargin=0.4 cm + 10pt
    ]
}{
    \end{itemize}
} % new environment for highlights

\newenvironment{highlightsforbulletentries}{
    \begin{itemize}[
        topsep=0.10 cm,
        parsep=0.10 cm,
        partopsep=0pt,
        itemsep=0pt,
        leftmargin=10pt
    ]
}{
    \end{itemize}
} % new environment for highlights for bullet entries


\newenvironment{onecolentry}{
    \begin{adjustwidth}{
        0.2 cm + 0.00001 cm
    }{
        0.2 cm + 0.00001 cm
    }
}{
    \end{adjustwidth}
} % new environment for one column entries

\newenvironment{twocolentry}[2][]{
    \onecolentry
    \def\secondColumn{#2}
    \setcolumnwidth{\fill, 4.5 cm}
    \begin{paracol}{2}
}{
    \switchcolumn \raggedleft \secondColumn
    \end{paracol}
    \endonecolentry
} % new environment for two column entries

\newenvironment{header}{
    \setlength{\topsep}{0pt}\par\kern\topsep\centering\linespread{1.5}
}{
    \par\kern\topsep
} % new environment for the header

\newcommand{\placelastupdatedtext}{% \placetextbox{<horizontal pos>}{<vertical pos>}{<stuff>}
  \AddToShipoutPictureFG*{% Add <stuff> to current page foreground
    \put(
        \LenToUnit{\paperwidth-2 cm-0.2 cm+0.05cm},
        \LenToUnit{\paperheight-1.0 cm}
    ){\vtop{{\null}\makebox[0pt][c]{
        \small\color{gray}\textit{Last updated in January 2025}\hspace{\widthof{Last updated in January 2025}}
    }}}%
  }%
}%

% save the original href command in a new command:
\let\hrefWithoutArrow\href

% new command for external links:
\renewcommand{\href}[2]{\hrefWithoutArrow{#1}{\ifthenelse{\equal{#2}{}}{ }{#2 }\raisebox{.15ex}{\footnotesize \faExternalLink*}}}


\begin{document}
    \newcommand{\AND}{\unskip
        \cleaders\copy\ANDbox\hskip\wd\ANDbox
        \ignorespaces
    }
    \newsavebox\ANDbox
    \sbox\ANDbox{}

    \placelastupdatedtext
    \begin{header}
        \textbf{\fontsize{24 pt}{24 pt}\selectfont Elijah Parker}

        \vspace{0.3 cm}

        \normalsize
        \mbox{{\color{black}\footnotesize\faMapMarker*}\hspace*{0.13cm}Laurel, MD}%
        \kern 0.25 cm%
        \AND%
        \kern 0.25 cm%
        \mbox{\hrefWithoutArrow{mailto:elijahparker000@gmail.com}{\color{black}{\footnotesize\faEnvelope[regular]}\hspace*{0.13cm}elijahparker000@gmail.com}}%
        \kern 0.25 cm%
        \AND%
        \kern 0.25 cm%
        \mbox{\hrefWithoutArrow{tel:+1-256-996-5241}{\color{black}{\footnotesize\faPhone*}\hspace*{0.13cm}(256) 996-5241}}%
        \kern 0.25 cm%
        \AND%
        \kern 0.25 cm%
        \mbox{\hrefWithoutArrow{https://elijahparker000.com/}{\color{black}{\footnotesize\faLink}\hspace*{0.13cm}elijahparker000.com}}%
        \kern 0.25 cm%
        \AND%
        \kern 0.25 cm%
        \mbox{\hrefWithoutArrow{https://linkedin.com/in/elijahparker000}{\color{black}{\footnotesize\faLinkedinIn}\hspace*{0.13cm}elijahparker000}}%
        \kern 0.25 cm%
        \AND%
        \kern 0.25 cm%
        \mbox{\hrefWithoutArrow{https://github.com/elijahparker000}{\color{black}{\footnotesize\faGithub}\hspace*{0.13cm}elijahparker000}}%
    \end{header}

    \vspace{0.3 cm - 0.3 cm}


    \section{Welcome to RenderCV!}
        \begin{onecolentry}
            \href{https://rendercv.com}{RenderCV} is a LaTeX-based CV/resume version-control and maintenance app. It allows you to create a high-quality CV or resume as a PDF file from a YAML file, with \textbf{Markdown syntax support} and \textbf{complete control over the LaTeX code}.
        \end{onecolentry}

        \vspace{0.2 cm}

        \begin{onecolentry}
            The boilerplate content was inspired by \href{https://github.com/dnl-blkv/mcdowell-cv}{Gayle McDowell}.
        \end{onecolentry}
    \section{Quick Guide}

    \begin{onecolentry}
        \begin{highlightsforbulletentries}


        \item Each section title is arbitrary and each section contains a list of entries.

        \item There are 7 unique entry types: \textit{BulletEntry}, \textit{TextEntry}, \textit{EducationEntry}, \textit{ExperienceEntry}, \textit{NormalEntry}, \textit{PublicationEntry}, and \textit{OneLineEntry}.

        \item Select a section title, pick an entry type, and start writing your section!

        \item \href{https://docs.rendercv.com/user_guide/}{Here}, you can find a comprehensive user guide for RenderCV.


        \end{highlightsforbulletentries}
    \end{onecolentry}

    \section{Education}
        \begin{twocolentry}{
        \textit{August 2024 – Present}}
            \textbf{Johns Hopkins University}

            \textit{MS in Artificial Intelligence}
        \end{twocolentry}

        \vspace{0.10 cm}
        \begin{onecolentry}
            \begin{highlights}
                \item GPA: 4.0/4.0 %(\href{https://example.com}{a link to somewhere})
                \item \textbf{Coursework:} Algorithms, \LaTeX
            \end{highlights}
        \end{onecolentry}

        \vspace{0.2 cm}

        \begin{twocolentry}{
        \textit{August 2019 – May 2024}}
            \textbf{Auburn University}

            \textit{BS in Electrical Engineering}
            \newline
            \textit{BS in Computer Engineering}
        \end{twocolentry}

        \vspace{0.10 cm}
        \begin{onecolentry}
            \begin{highlights}
                \item GPA: 4.0/4.0 %(\href{https://example.com}{a link to somewhere})
                \item \textbf{Coursework:} Computer Architecture, Electromagnetism, Algorithms, Random Signals and Systems, Digital Electronics, Analog Electronics, Control Systems, Electrical Power Engineering, Discrete Structures
            \end{highlights}
        \end{onecolentry}








    \section{Experience}



        
        \begin{twocolentry}{
        \textit{Laurel, MD}    
            
        \textit{June 2024 – Present}}
            \textbf{Wireless Communications Engineer}
            
            \textit{Johns Hopkins Applied Physics Laboratory}
        \end{twocolentry}

        \vspace{0.10 cm}
        \begin{onecolentry}
            \begin{highlights}
                \item Software Development, C++, Cellular Communications, Wireless and Wired Networks, OpenVPN, Python, GUI Development, REST APIs
            \end{highlights}
        \end{onecolentry}

        \vspace{0.2 cm}

        \begin{twocolentry}{
        \textit{Auburn, AL}    
            
        \textit{October 2023 – April 2024}}
            \textbf{Undergraduate AI Researcher}
            
            \textit{Open-Ended Reasoning and Knowledge Acquisition Lab}
        \end{twocolentry}

        \vspace{0.10 cm}
        \begin{onecolentry}
            \begin{highlights}
                \item Implemented MFCC technique for speech recognition to extract audio features from the EPIC-KITCHENS-100 dataset for multi-modal understanding in ego-centric videos
            \end{highlights}
        \end{onecolentry}

        \vspace{0.2 cm}

        \begin{twocolentry}{
        \textit{Auburn, AL}    
            
        \textit{February 2023 – May 2024}}
            \textbf{Student Employee}
            
            \textit{Alabama Micro/Nano Fabrication Lab}
        \end{twocolentry}

        \vspace{0.10 cm}
        \begin{onecolentry}
            \begin{highlights}
                \item Aided electrical engineering PhD students with chemical processes related to microelectronics fabrication
                \item Designed and manufactureed custom aluminum parts with Fusion 360 and Haas CNC
                \item Maintained and documented Python code to streamline process of taking inventory of chemicals
                \item Cleaned, organized, and maintained cleanrooms, chemical labs, and equipment
            \end{highlights}
        \end{onecolentry}

        \vspace{0.2 cm}

        \begin{twocolentry}{
        \textit{Auburn, AL}    
            
        \textit{October 2019 – May 2024}}
            \textbf{Student Employee}

            \textit{Auburn University Electrical and Computer Engineering Department}
        \end{twocolentry}

        \vspace{0.10 cm}
        \begin{onecolentry}
            \begin{highlights}
                \item Aided students and faculty throughout Electrical and Computer Engineering Department especially concerning production of senior design projects
                \item Organized, cleaned, and maintained electronics and equipment
            \end{highlights}
        \end{onecolentry}





    
    \section{Publications}



        
        \begin{samepage}
            \begin{twocolentry}{
                Jan 2004
            }
                \textbf{3D Finite Element Analysis of No-Insulation Coils}

                \vspace{0.10 cm}

                \mbox{Frodo Baggins}, \mbox{\textbf{\textit{Elijah Parker}}}, \mbox{Samwise Gamgee}
            \end{twocolentry}


            \vspace{0.10 cm}

            \begin{onecolentry}
        \href{https://doi.org/10.1109/TASC.2023.3340648}{10.1109/TASC.2023.3340648}
            \end{onecolentry}
        \end{samepage}


    
    \section{Projects}



        
        \begin{twocolentry}{
            
            
        \textit{\href{https://github.com/sinaatalay/rendercv}{github.com/name/repo}}}
            \textbf{Multi-User Drawing Tool}
        \end{twocolentry}

        \vspace{0.10 cm}
        \begin{onecolentry}
            \begin{highlights}
                \item Developed an electronic classroom where multiple users can simultaneously view and draw on a "chalkboard" with each person's edits synchronized
                \item Tools Used: C++, MFC
            \end{highlights}
        \end{onecolentry}


        \vspace{0.2 cm}

        \begin{twocolentry}{
            
            
        \textit{\href{https://github.com/sinaatalay/rendercv}{github.com/name/repo}}}
            \textbf{Synchronized Desktop Calendar}
        \end{twocolentry}

        \vspace{0.10 cm}
        \begin{onecolentry}
            \begin{highlights}
                \item Developed a desktop calendar with globally shared and synchronized calendars, allowing users to schedule meetings with other users
                \item Tools Used: C\#, .NET, SQL, XML
            \end{highlights}
        \end{onecolentry}


        \vspace{0.2 cm}

        \begin{twocolentry}{
            
            
        \textit{2002}}
            \textbf{Custom Operating System}
        \end{twocolentry}

        \vspace{0.10 cm}
        \begin{onecolentry}
            \begin{highlights}
                \item Built a UNIX-style OS with a scheduler, file system, text editor, and calculator
                \item Tools Used: C
            \end{highlights}
        \end{onecolentry}



    
    \section{Technologies}



        
        \begin{onecolentry}
            \textbf{Languages:} C++, C, Java, Objective-C, C\#, SQL, JavaScript
        \end{onecolentry}

        \vspace{0.2 cm}

        \begin{onecolentry}
            \textbf{Technologies:} .NET, Microsoft SQL Server, XCode, Interface Builder
        \end{onecolentry}


    

\end{document}